\documentclass{article}
\usepackage[a4paper,top=0.75in, bottom=0.75in, left=1in, right=1in,footskip=0.2in]{geometry}
%\usepackage{fullpage}
\usepackage{listings}
\usepackage{gensymb}
\usepackage{hyperref}
\hypersetup{
	 colorlinks   = true,
     citecolor    = black,
     linkcolor    = black,
     urlcolor     = black
}
\usepackage{graphicx}
\usepackage{algorithm}
\usepackage{algpseudocode}
\usepackage{amsmath}
\usepackage{amssymb}
\usepackage{tikz}
\usepackage{caption}
\usepackage{subcaption}
\usepackage{float}
\usetikzlibrary{arrows,matrix,positioning}
\setcounter{tocdepth}{3}
\begin{document}
\title{DIP Homework 4}
\author{Qiuyi Zhang 12330402 \\ \href{mailto:joyeec9h3@gmail.com}{joyeec9h3@gmail.com}} 
\date{\today}
\maketitle
\tableofcontents
\section{Exercises}

\subsection{Color Spaces}

\textbf{Answer:} 

\begin{enumerate}
\item Advantages of HSI color space:
\begin{enumerate}
\item xxx
\item xxx
\end{enumerate}

\item Adding $120\degree$ to the Hue components
\end{enumerate}

\subsection{Color Composition}

\textbf{Answer:}

General expression


% -------------------- Programming Tasks ------------------------
\section{Programming Tasks}
% -------------------- Fourier Transform ------------------------
\subsection{Image Filtering}
% -------------------- Results ------------------------

\begin{figure}[H]
	\centering
	% pt = px * 72 / DPI
	\includegraphics[width=192pt]{../img/task_1.png}
	\caption{The original image}
\end{figure}

\subsubsection{Arithmetic mean filter}

The images filtered with $3 \times 3$ and $9 \times 9$ arithmetic mean filters are shown in Figure~\ref{fig:baram33} and~\ref{fig:baram99}.

Both filters make the bars appear to be ``smaller'' in both height and width. Since the white sides of the edges will take the mean of their neighborhood as their new intensities, they will be grey instead of white after filtering, thus visually ``shrinking'' the bars. This effect also darkens the overall color of the image.

\begin{figure}[H]
	\captionsetup{justification=centering,margin=1cm}
	\begin{minipage}[b]{0.48\linewidth}
		\centering
		% pt = px * 72 / DPI
		\includegraphics[width=192pt]{../result/task1/arithmetic-mean-3-3.png}
		\caption{Filtered with $3 \times 3$ arithmetic mean filter}
		\label{fig:baram33}
	\end{minipage}
	\begin{minipage}[b]{0.48\linewidth}
		\centering
		% pt = px * 72 / DPI
		\includegraphics[width=192pt]{../result/task1/arithmetic-mean-9-9.png}
		\caption{Filtered with $9 \times 9$ arithmetic mean filter}
		\label{fig:baram99}
	\end{minipage}
\end{figure}


\subsubsection{Harmonic mean filter}

The images filtered with $3 \times 3$ and $9 \times 9$ harmonic mean filters are shown in Figure~\ref{fig:barhm33} and~\ref{fig:barhm99}.

The expression of harmonic mean filtering is:

$$
\hat{f}(x, y) = \frac{mn}{\sum_{(s, t)\in S_{xy}} \frac{1}{g(s, t)}}
$$

Once there is any $g(s, t) = 0$, then $\sum_{(s, t)\in S_{xy}} \frac{1}{g(s, t)} = \infty$, then $\hat{f}(x, y) = 0$. Assuming $n$ is odd, $n = 2k + 1$ and $2k <$ the distance between the bars, an $n \times n$ harmonic mean filter will ``blacken'' $k$ pixels of the edges of the bars in every direction. Therefore with a $3 \times 3$ harmonic filter, the bars are $2 \times 1 = 2$ pixels thiner, $2 \times 1 = 2$ pixels shorter. When the neighborhood is $9 \times 9$, the bars are $4 \times 1 = 4$ pixels thiner, $4 \times 1 = 4$ pixels shorter, which effectively makes the bars disappear.


\begin{figure}[H]
	\captionsetup{justification=centering,margin=1cm}
	\begin{minipage}[b]{0.48\linewidth}
		\centering
		% pt = px * 72 / DPI
		\includegraphics[width=192pt]{../result/task1/harmonic-mean-3-3.png}
		\caption{Filtered with $3 \times 3$ harmonic mean filter}
		\label{fig:barhm33}
	\end{minipage}
	\begin{minipage}[b]{0.48\linewidth}
		\centering
		% pt = px * 72 / DPI
		\includegraphics[width=192pt]{../result/task1/harmonic-mean-9-9.png}
		\caption{Filtered with $9 \times 9$ harmonic mean filter}
		\label{fig:barhm99}
	\end{minipage}
\end{figure}

\subsubsection{Contraharmonic mean filter}

The images filtered with $3 \times 3$ and $9 \times 9$ contraharmonic mean filters are shown in Figure~\ref{fig:barchm33} and~\ref{fig:barchm99}

The expression of contraharmonic mean filtering is:

$$
\hat{f}(x, y) = \frac{\sum_{(s, t)\in S_{xy}} g(s, t)^{Q+1}}{\sum_{(s, t)\in S_{xy}} g(s, t)^Q}
$$

When $Q < 0$, once there is any $g(s, t) = 0$, then $\sum_{(s, t)\in S_{xy}} g(s, t)^{Q} = 0$, then $\hat{f}(x, y) = 0$. When $Q = -1.5$, using the same reasoning as for the harmonic mean filter, a $3 \times 3$ contraharmonic filter will make the bars $2 \times 1 = 2$ pixels thiner, $2 \times 1 = 2$ pixels shorter. A $9 \times 9$ contraharmonic filter will make the bars $4 \times 1 = 4$ pixels thiner, $4 \times 1 = 4$ pixels shorter, effectively making the bars disappear.

\begin{figure}[H]
	\captionsetup{justification=centering,margin=0.5cm}
	\begin{minipage}[b]{0.48\linewidth}
		\centering
		% pt = px * 72 / DPI
		\includegraphics[width=192pt]{../result/task1/contraharmonic-mean-3-3.png}
		\caption{Filtered with $3 \times 3$ contraharmonic mean filter($Q=-1.5$)}
		\label{fig:barchm33}
	\end{minipage}
	\begin{minipage}[b]{0.48\linewidth}
		\centering
		% pt = px * 72 / DPI
		\includegraphics[width=192pt]{../result/task1/contraharmonic-mean-9-9.png}
		\caption{Filtered with $9 \times 9$ contraharmonic mean filter($Q=-1.5$)}
		\label{fig:barchm99}
	\end{minipage}
\end{figure}

% End results and descirptions


\subsection{Image Denoising}

\subsubsection{Statistical filters}

\paragraph{Algorithm}
Discuss how you implement this operation in less than 1 page.

\begin{figure}[]
	\centering
	% pt = px * 72 / DPI
	\includegraphics[width=336pt]{../img/task_2.png}
	\caption{The original image}
\end{figure}

\subsubsection{Gaussian noise and denoising}

\paragraph{Results}
\begin{figure}[H]
	\centering
	% pt = px * 72 / DPI
	\includegraphics[width=336pt]{../result/task2/gauss/gauss-0-40.png}
	\caption{Image with gaussian noise($\mu = 0, \sigma = 40$)}
	\label{fig:gauss}
\end{figure}

\begin{figure}[H]
	\centering
	% pt = px * 72 / DPI
	\includegraphics[width=336pt]{../result/task2/gauss/gauss-arithmetic.png}
	\caption{Gaussian noise filtered with $3 \times 3$ arithmetic mean filter}
	\label{fig:gaussam}
\end{figure}

\begin{figure}[H]
	\centering
	% pt = px * 72 / DPI
	\includegraphics[width=336pt]{../result/task2/gauss/gauss-geometric.png}
	\caption{Gaussian noise filtered with $3 \times 3$ geometric mean filter}
	\label{fig:gaussgm}
\end{figure}

\begin{figure}[H]
	\centering
	% pt = px * 72 / DPI
	\includegraphics[width=336pt]{../result/task2/gauss/gauss-harmonic.png}
	\caption{Gaussian noise filtered with $3 \times 3$ harmonic mean filter}
	\label{fig:gausshm}
\end{figure}

\begin{figure}[H]
	\centering
	% pt = px * 72 / DPI
	\includegraphics[width=336pt]{../result/task2/gauss/gauss-contraharmonic.png}
	\caption{Gaussian noise filtered with $3 \times 3$ contraharmonic mean filter($Q = -1.5$)}
	\label{fig:gausschm}
\end{figure}

\begin{figure}[H]
	\centering
	% pt = px * 72 / DPI
	\includegraphics[width=336pt]{../result/task2/gauss/gauss-median.png}
	\caption{Gaussian noise filtered with $3 \times 3$ median filter}
	\label{fig:gaussm}
\end{figure}


\paragraph{Discussion}
Compare these results, and discuss which one looks better / worse, and why, within 1 page

\subsubsection[H]{Salt noise and denoising}

\paragraph{Results}

\begin{figure}[H]
	\centering
	% pt = px * 72 / DPI
	\includegraphics[width=336pt]{../result/task2/salt/salt-20.png}
	\caption{Image with salt noise($p=0.2$)}
	\label{fig:salt}
\end{figure}

\begin{figure}[H]
	\centering
	% pt = px * 72 / DPI
	\includegraphics[width=336pt]{../result/task2/salt/salt-contraharmonic-1-5.png}
	\caption{Salt noise filtered with $3 \times 3$ contraharmonic mean filter($Q = 1.5$)}
	\label{fig:saltchmpos}
\end{figure}

\begin{figure}[H]
	\centering
	% pt = px * 72 / DPI
	\includegraphics[width=336pt]{../result/task2/salt/salt-contraharmonic--1-5.png}
	\caption{Salt noise filtered with $3 \times 3$ contraharmonic mean filter($Q = -1.5$)}
	\label{fig:saltchmneg}
\end{figure}

\paragraph{Discussion}
Discuss why setting a wrong value for Q would lead to terrible results within 1 page.

\subsubsection{Salt-and-pepper noise and denoising}
\paragraph{Results}


\begin{figure}[H]
	\centering
	% pt = px * 72 / DPI
	\includegraphics[width=336pt]{../result/task2/sap/sap-20-20.png}
	\caption{Image with salt noise($p=0.2$) and pepper noise($p=0.2$)}
	\label{fig:sap}
\end{figure}

\begin{figure}[H]
	\centering
	% pt = px * 72 / DPI
	\includegraphics[width=336pt]{../result/task2/sap/sap-arithmetic.png}
	\caption{Salt-and-pepper noise filtered with $3 \times 3$ arithmetic mean filter}
	\label{fig:sapam}
\end{figure}

\begin{figure}[H]
	\centering
	% pt = px * 72 / DPI
	\includegraphics[width=336pt]{../result/task2/sap/sap-harmonic.png}
	\caption{Salt-and-pepper filtered with $3 \times 3$ harmonic mean filter}
	\label{fig:saphm}
\end{figure}


\begin{figure}[H]
	\centering
	% pt = px * 72 / DPI
	\includegraphics[width=336pt]{../result/task2/sap/sap-contraharmonic.png}
	\caption{Salt-and-pepper noise filtered with $3 \times 3$ contraharmonic mean filter($Q=0.5$)}
	\label{fig:sapchm}
\end{figure}

\begin{figure}[H]
	\centering
	% pt = px * 72 / DPI
	\includegraphics[width=336pt]{../result/task2/sap/sap-max.png}
	\caption{Salt-and-pepper noise filtered with $3 \times 3$ max filter}
	\label{fig:sapmax}
\end{figure}

\begin{figure}[H]
	\centering
	% pt = px * 72 / DPI
	\includegraphics[width=336pt]{../result/task2/sap/sap-min.png}
	\caption{Salt-and-pepper noise filtered with $3 \times 3$ min filter}
	\label{fig:sapmin}
\end{figure}

\begin{figure}[H]
	\centering
	% pt = px * 72 / DPI
	\includegraphics[width=336pt]{../result/task2/sap/sap-median.png}
	\caption{Salt-and-pepper noise filtered with $3 \times 3$ median filter}
	\label{fig:sapmedian}
\end{figure}

\paragraph{Discussion}
Compare these results, and discuss which one looks better / worse, and why, within 1 page

\subsection{Histogram Equalization on Color Images}

\subsubsection{Results}

\begin{figure}[H]
	\centering
	% pt = px * 72 / DPI
	\includegraphics[width=288pt]{../img/02.png}
	\caption{The original image}
\end{figure}

\begin{figure}[H]
	\centering
	% pt = px * 72 / DPI
	\includegraphics[width=288pt]{../result/hist/hist-seperate.png}
	\caption{Image rebuilt with three channels equalized with different histograms}
\end{figure}

\begin{figure}[h]
	\centering
	% pt = px * 72 / DPI
	\includegraphics[width=288pt]{../result/hist/hist-together.png}
	\caption{Image rebuilt with three channels equalized with the same average histogram}
\end{figure}


\end{document}